% \iffalse
\let\negmedspace\undefined
\let\negthickspace\undefined
\documentclass[journal,12pt,twocolumn]{IEEEtran}
\usepackage{cite}
\usepackage{amsmath,amssymb,amsfonts,amsthm}
\usepackage{algorithmic}
\usepackage{graphicx}
\usepackage{textcomp}
\usepackage{xcolor}
\usepackage{txfonts}
\usepackage{listings}
\usepackage{enumitem}
\usepackage{mathtools}
\usepackage{gensymb}
\usepackage{comment}
\usepackage[breaklinks=true]{hyperref}
\usepackage{tkz-euclide} 
\usepackage{listings}
\usepackage{gvv}                                        
\def\inputGnumericTable{}                                 
\usepackage[latin1]{inputenc}                                
\usepackage{color}                                            
\usepackage{array}                                            
\usepackage{longtable}                                       
\usepackage{calc}                                             
\usepackage{multirow}                                         
\usepackage{hhline}                                           
\usepackage{ifthen}                                           
\usepackage{lscape}
\newtheorem{theorem}{Theorem}[section]
\newtheorem{problem}{Problem}
\newtheorem{proposition}{Proposition}[section]
\newtheorem{lemma}{Lemma}[section]
\newtheorem{corollary}[theorem]{Corollary}
\newtheorem{example}{Example}[section]
\newtheorem{definition}[problem]{Definition}
\newcommand{\BEQA}{\begin{eqnarray}}
\newcommand{\EEQA}{\end{eqnarray}}
\newcommand{\define}{\stackrel{\triangle}{=}}
\theoremstyle{remark}
\newtheorem{rem}{Remark}
\begin{document}

\bibliographystyle{IEEEtran}
\vspace{3cm}
\title{NCERT Question 11.9.3.9}
\author{EE23BTECH11019 - Faisal Imtiyaz $^{*}$% <-this % stops a space
}
\maketitle
\newpage
\bigskip

\renewcommand{\thefigure}{\arabic{figure}}
\renewcommand{\thetable}{\arabic{table}}


\vspace{3cm}
\textbf{Question:} Find the sum to indicated number of terms in the geometric progression:\\
$1,-a, a^2, -a^3,...n$ terms (if $a\neq-1$).\\
\solution
\begin{table}[htbp]
    \centering
    \def\arraystrech{1.5}
        \begin{tabular}{|p{2.5cm}|p{1.5cm}|p{3cm}|}
    \hline
            \textbf{Input Parameters} & \textbf{Values} & \textbf{Description} \\
    \hline
            $x(0)$ & $1$ & First term\\
    \hline
            $r$ & $(-a)$ & Common ratio\\
    \hline
            $x(n)$ & $(-a)^{n}u(n)$ & General term \\
    \hline
    \end{tabular}
    \caption{Given inputs}
    \label{tab:1.11.3.9}
\end{table} 
    
\newline
From \tabref{tab:1.11.3.9},
\begin{align}
X\brak{z} =& \frac{1}{1+az^{-1}}\\
y(n) =& \sum_{k=0}^{n} (-a)^k = \sum_{k=-\infty}^{n} (-a)^k u(k) \\
y\brak{n}= &{(-a)^n}u(n)*u\brak{n}\\
\implies Y\brak{z}= &X\brak{z}\cdot\ U\brak{z}
\end{align}
\begin{align}
    Y\brak{z} = &\frac{1}{1+az^{-1}}\cdot\frac{1}{1-z^{-1}}\\
    \implies Y\brak{z} = &\frac{z^{2}}{(z+a)(z-1)}
\end{align}
Using Z transform pairs  to find the inverse Z-transform:\\
\begin{align}
    Y\brak{z}=& \frac{z^{2}}{a+1}\sbrak{\frac{1}{z-1} - \frac{1}{z+a}}\\
    =& \frac{1}{a+1}\sbrak{\frac{z^{2}-1}{z-1}+\frac{1}{z-1} -\frac{z^{2}-a^{2}}{z+a} -\frac{a^{2}}{z+a}} \\
    =& \frac{1}{a+1}\sbrak{(z-1)+\frac{1}{z-1}-(z-a)-\frac{a^{2}}{z+a}}\\
    =& 1+\frac{1}{a+1}\sbrak{\frac{1}{z-1}-\frac{a^{2}}{z+a}}\\
    y\brak{n} =& \delta\brak{n} + \frac{1}{a+1}\sbrak{1- a^{2}\cdot(-a)^{n}}\\
    y\brak{n} =& \delta\brak{n}+ \frac{1-(-a)^{n}}{1-(-a)}\\
\end{align}
Since $\delta\brak{n}$ is zero for $n>0$, thus:
\begin{align}
    y\brak{n}=\frac{1-(-a)^{n}}{1-(-a)}\\
\end{align}

\end{document}
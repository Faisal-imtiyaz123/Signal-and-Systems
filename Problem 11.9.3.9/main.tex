% \iffalse
\let\negmedspace\undefined
\let\negthickspace\undefined
\documentclass[journal,12pt,twocolumn]{IEEEtran}
\usepackage{cite}
\usepackage{amsmath,amssymb,amsfonts,amsthm}
\usepackage{algorithmic}
\usepackage{graphicx}
\usepackage{textcomp}
\usepackage{xcolor}
\usepackage{txfonts}
\usepackage{listings}
\usepackage{enumitem}
\usepackage{mathtools}
\usepackage{gensymb}
\usepackage{comment}
\usepackage[breaklinks=true]{hyperref}
\usepackage{tkz-euclide} 
\usepackage{listings}
\usepackage{gvv}                                        
\def\inputGnumericTable{}                                 
\usepackage[latin1]{inputenc}                                
\usepackage{color}                                            
\usepackage{array}                                            
\usepackage{longtable}                                       
\usepackage{calc}                                             
\usepackage{multirow}                                         
\usepackage{hhline}                                           
\usepackage{ifthen}                                           
\usepackage{lscape}
\newtheorem{theorem}{Theorem}[section]
\newtheorem{problem}{Problem}
\newtheorem{proposition}{Proposition}[section]
\newtheorem{lemma}{Lemma}[section]
\newtheorem{corollary}[theorem]{Corollary}
\newtheorem{example}{Example}[section]
\newtheorem{definition}[problem]{Definition}
\newcommand{\BEQA}{\begin{eqnarray}}
\newcommand{\EEQA}{\end{eqnarray}}
\newcommand{\define}{\stackrel{\triangle}{=}}
\theoremstyle{remark}
\newtheorem{rem}{Remark}
\begin{document}

\bibliographystyle{IEEEtran}
\vspace{3cm}
\title{NCERT Question 11.9.3.9}
\author{EE23BTECH11019 - Faisal Imtiyaz $^{*}$% <-this % stops a space
}
\maketitle
\newpage
\bigskip

\renewcommand{\thefigure}{\arabic{figure}}
\renewcommand{\thetable}{\arabic{table}}


\vspace{3cm}
\textbf{Question:} Find the sum to indicated number of terms in the geometric progression:\\
$1,-a, a^2, -a^3,...n$ terms (if $a\neq-1$).\\
\solution
\input{tables/11.9.3.9.table1}
% \newline
\begin{table}[htbp]
    \centering
    \def\arraystrech{1.5}
        \begin{tabular}{|p{2.5cm}|p{1.5cm}|p{3cm}|}
    \hline
            \textbf{Signal} & \textbf{Transform} \\
    \hline
            $ \frac{1}{1-z^{-1}} $ & $u(n)$ & \\
    \hline
            $ \frac{1}{1-az^{-1}}$ & $(a)^{n}u(n)$ \\
    \hline
    \end{tabular}
    \caption{Given inputs}
    \label{pairs}
\end{table} 
    
Poles of the respecttive signals are as:
\begin{align}
    1-z^{-1} =& 0\\ 
    \implies  |z|  =1\\
    1-az^{-1} =& 0\\
    \implies |z| = \frac{1}{a}\\
\end{align}
Both the Signals are right sided so ROC is :
\begin{align}
    |z| > 1 \text{and} |z| > \frac{1}{a}\\
\end{align}
So for $0<a<1$ ROC is $|z| > \frac{1}{a}$ and otherwise ROC is $|z| > 1$\\
From \tabref{tab:1.11.3.9},
\begin{align}
X\brak{z} =& \frac{1}{1+az^{-1}}\\
y\brak{n}= &{(-a)^n}u(n)*u\brak{n}\\
\implies Y\brak{z}= &X\brak{z}\cdot\ U\brak{z}
\end{align}
\begin{align}
    = &\frac{1}{1+az^{-1}}\cdot\frac{1}{1-z^{-1}}\\
\end{align}
Using Z transform pairs  to find the inverse Z-transform:\\
\begin{align}
    Y\brak{z}=& \frac{1}{a+1}\sbrak{\frac{1}{1-z^{-1}} - \frac{1}{1+az^{-1}}}\\
    =& \frac{1}{a+1}\sbrak{\frac{1-z^{-2}}{1-z^{-1}}+\frac{z^{-1}}{1-z^{-1}} -\frac{{1}-a^{2}z^{-1}}{1+az^{-1}} -\frac{a^{2}z^{-1}}{1+az^{-1}}}\\
    =& 1+\frac{1}{a+1}\sbrak{\frac{z^{-1}}{1-z^{-1}}-\frac{a^{2}z^{-1}}{1+az^{-1}}}\\
    y\brak{n} =& \delta\brak{n} + \sbrak{\frac{1}{a+1}\sbrak{1- a^{2}\cdot(-a)^{n}}}u\brak{n}\\
    y\brak{n} =& \delta\brak{n}+ \sbrak{\frac{1-(-a)^{n}}{1-(-a)}}u\brak{n}
\end{align}
\begin{figure}[ht!]
	\includegraphics[width=\columnwidth]{plots/plot-file.png}
	\caption{Plot of $y(n)$}
	\label{fig:1.2}
\end{figure}
\end{document}
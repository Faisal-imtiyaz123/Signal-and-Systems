\documentclass[journal,12pt,twocolumn]{IEEEtran}

\usepackage{cite}
\usepackage{tikz}
\usetikzlibrary{angles, quotes}
\usepackage{amsmath,amssymb,amsfonts,amsthm}
\usepackage{algorithmic}
\usepackage{graphicx}
\usepackage{textcomp}
\usepackage{xcolor}
\usepackage{txfonts}
\usepackage{listings}
\usepackage{enumitem}
\usepackage{mathtools}
\usepackage{gensymb}
\usepackage{comment}
\usepackage[breaklinks=true]{hyperref}
\usepackage{tkz-euclide}
\usepackage{gvv}
\def\inputGnumericTable{}
\usepackage[latin1]{inputenc}
\usepackage{color}
\usepackage{array}
\usepackage{longtable}
\usepackage{calc}
\usepackage{multirow}
\usepackage{hhline}
\usepackage{ifthen}
\usepackage{lscape}

\newtheorem{theorem}{Theorem}[section]
\newtheorem{problem}{Problem}
\newtheorem{proposition}{Proposition}[section]
\newtheorem{lemma}{Lemma}[section]
\newtheorem{corollary}[theorem]{Corollary}
\newtheorem{example}{Example}[section]
\newtheorem{definition}[problem]{Definition}
\theoremstyle{remark}
\newtheorem{rem}{Remark}

\begin{document}

\title{ANALOG: 12.10.19}
\author{EE23BTECH11019 - Faisal Imtiyaz$^{*}$}% <-this % stops a space
\maketitle

\section{Problem Statement}

A parallel beam of light with a wavelength of $500$ nm falls on a narrow slit, and the resulting diffraction pattern is observed on a screen $1$ m away. The distance to the first minimum from the center of the screen is $2.5$ mm.

Find the width of the slit given that $y = 0.0025$ m, $L = 1$ m, and $\lambda = 5 \times 10^{-7}$ m.\\


\section{Derivation}

\begin{tikzpicture}
\draw[<->] (-0.5,-1) -- node[left] {$d$} (-0.5,1);
\draw[<->] (-0.2,-0.1) -- node[right] {$s$} (-0.2,1);
\draw (0,-6)--(0,-1) (0,1)--(0,6);
\draw (7,-6) -- (7,6);
\node at (4,0.1) [above] {$x+dx$};
\node at (4,1.5) [above] {$x$};
\draw[gray] (0,-0.1) -- (7,2);
\draw[gray] (0,1) -- (7,2);
\draw[thick] (0,-0.1) -- (0,0.1);
\coordinate[label=right:$P$] (P) at (7,2);
\coordinate[label=right:$R$] (R) at (2,-0.1);
\coordinate[label=below:$ds$] (S) at (0,-0.1);
\fill[black] (0,-0.2) rectangle (0.08,0.1);
\pic[draw, angle radius=1cm, angle eccentricity=1.5, "$\theta$"] {angle = R--S--P};
\draw[thick,dotted] (0,-0.1)--(2,-0.1);
\draw[thick,dotted] (0,1) -- (0.5,0.1);
\end{tikzpicture}

Let $dE$ be the electric field at point $P$ due to light from $ds$ part of the slit. Since the Electric field $dE$ is proportional to the small slit width $ds$, we can say that:
\begin{align*}
dE &= G \cos(k(x+dx) - \omega t) \, ds \\
&\implies E = G \int_{0}^{b} \cos(k(x+dx) - \omega t) \, ds
\end{align*}

Here, $G$ is the constant of proportionality.

Now, from the figure it can be easily seen that $s\sin\theta = dx$, so,
\begin{align*}
E &= G \int_{0}^{b} \cos(k(x+s\sin\theta) - \omega t) \, ds \\
&= G \left[\frac{\sin((k \sin\theta) s + (kx - \omega t)s)}{k \sin\theta}\right]_{s=0}^{s=b} \\
&= G\left[\frac{\sin(kb\sin\theta+kx-wt)-\sin(kx-wt)}{k\sin\theta}\right] \\
&= \frac{2G}{k\sin\theta}\sin\left(\frac{kb\sin\theta}{2}\right)\cos\left(kx-wt+\frac{kb\sin\theta}{2}\right)
\end{align*}


Now, let $\frac{k\beta\sin\theta}{2} = \beta$. Also, the above equation represents a wave with amplitude $A$ of $\frac{2G}{k\sin\theta}\sin\left(\frac{k\beta\sin\theta}{2}\right)$.\\


$A$ can be written as $A = \frac{Gb\cdot\sin\beta}{\beta}$.
The intensity of the wave will be proportional to $A^2$. For minima, the intensity should be $0$. Therefore, for minima we have:
$$
\sin\beta = 0,  \beta \neq 0
$$
Therefore, $$
\sin\beta = n\pi
$$
where \(n\) is a natural number.

So, the first minima should occur at:
\begin{align*}
&\beta = \pi,\\
&\frac{kb\sin\theta}{2} = \pi,\\
&\frac{\pi\sin\theta}{\lambda} = \pi,\\
&\sin\theta = \frac{\lambda}{b}
\end{align*}

\section{Solution}

The first minimum is given by:
\[
\sin\theta = \frac{\lambda}{b}
\]

Where \(b\) is the width of the slit.

Now for a small angle \(\theta\), \(\sin\theta\) can be assumed to be equal to \(\theta\) as well as \(\tan\theta\). So we can say \(\sin\theta\) is approximately equal to \(\tan\theta\).

Therefore:
\[
\frac{\lambda}{b} = \frac{y}{L}
\]

Solving the expression we get:
\[
b = L \cdot \frac{\lambda}{y}
\]

Plugging in the values, \(b = 1 \times \frac{5 \times 10^{-7}}{0.0025} = 0.2\) mm.

\end{document}
